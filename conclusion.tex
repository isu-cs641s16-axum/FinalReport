\section{Conclusion and Future Work}
\label{sec:conclusion}

Because of the complexity of compiler toolchains, bugs in them are hard to detect and costly to debug. Hence it is advantageous to formally verify the correctness of a compiler to gain the confidence that programs written can be subjected to possible optimizations and desired result without violating expectations of some part of the systems' semantics.

In this work, we have designed a core calculus for Pig Latin which we believe models that language's five core features. On these foundations, we have provided a preliminary exploration on how one might use the Coq proof assistant to formally model Pig program compilation specifications---the first step to formally proving compilation correctness. The generality of this approach should help provide a way to formalize and verify the non-trivial mappings of properties between the various semantics of Pig, its intermediate representations, and MapReduce. To the best of our knowledge this is the first attempt to do any such formalization work of Pig Latin.

A future direction of this project would be to include the core calculus of MapReduce framework in context of Pig Latin compilation model and formalize the translation of Pig Latin programs to the MapReduce job generation stage. That would give a complete overview of the compilation algorithm of Pig programming model and formally verify the correctness of Pig programs.
