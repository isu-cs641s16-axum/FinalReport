\section{Conclusion and Future Work}
\label{sec:conclusion}

A program specified by a user may contain a number of bugs, mostly detectable by some written test cases. While logical and syntactical bugs in a program are easy to find, there is a possibility of bugs being introduced by a compiler - specially optimizing compilers, if the compiler is not in sync with the program specification and semantics and there are subtle differences. These bugs are hard to detect and costly to debug. Hence it is necessary to formally verify the correctness of a compiler to gain the confidence that programs written can be subjected to possible optimizations and desired result as per the program semantics. 
Our formal model of Pig program specifications and constraints over the relations provide a way to formalize the non trivial mapping of properties from Pig semantics to map-reduce semantics. We can also formally prove the the constraints are being satisfied when translating from one intermediate form to another. Our design of the core calculus also represent the five core features of the language. To the best of our knowledge this is the first attempt to formalize a language like Pig Latin with declarative syntax in imperative programming style. \\
The future direction of this project would be to include the core calculus of map-reduce framework in context of Pig Latin compilation model and formalize the translation of PigLatin programs till the map-reduce job generation stage. That would give a complete overview of the compilation algorithm of Pig programming model and formally verify the correctness of Pig programs.
