\section{Related Work}
\label{sec:related}

To the best of our knowledge this is the first attempt made to formalize the Pig Latin language model. Although there has been some prior works related to Coq formalization of relational database systems. Malecha et al\cite{malecha2010rdbms} implemented a lightweight fully verified relational database management system. They implemented functional specification of RDBMS behavior with proofs that the implementation meets the specification, all written in Coq. Their approach to model schema, relations and relational algebra uses denotational semantics. Their view towards relation is as a finite set of tuples over a list of primitive types and the list of types of the tuples in a relation is defined as a schema. Columns are identified by the offse in the schema similar to the approach we have taken. They selected a subset of SQL queries which can be supported by their defined Coq data types and presented an SQL-ish language. \\ 
Benzaken et al \cite{Benzaken2014} presented a Coq formalization of relational database systems focusing more on relational algebra and conjunctive queries. In contrast to the previous work mentioned, they chose to formalize the named SQL schema version. They provide a formal specification as well as a verified version of the algorithm that translate relational algebra to conjunctive queries. They have also modeled the functional and general dependencies between relations in the hope to achieve compiler optimization. \\
Leroy \cite{leroy2009formally} in his article described the development and formal verification of a compiler back-end using Coq proof assistant with the goal to prove the soundness of the compiler. Compiler introduced bugs are difficult to overcome and poses cost and threat to high assurance software. Normal testing method are not able to detect these bugs. Thus it is highly required for machine checked proof that compiler generated code behave exactly as per the specified semantics. Leroy et al reported in their work a formal verification of a lightly-optimizing compiler back-end that generates PowerPC assembly
code from a simple imperative intermediate language called Cminor using Coq Proof assistant.


