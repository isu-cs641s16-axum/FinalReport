\section{Introduction}
\label{sec:intro}

{\bf}

The exponential growth of sizes in data sets in this era of internet and data demands highly parallel, fault tolerant, and efficient systems. The MapReduce model \cite{dean2004mapreduce,dean2010mapreduce} has been widely for such for data manipulation. But, in spite of the appealing programming model, MapReduce imposes constraints to the flexibility and costly workarounds in expressing different data flow other than the two stage model mentioned above \cite{olston2008pig}.

Pig Latin comes into the picture to resolve these difficulties and provide a more imperative style to express relational queries. The programming model provides an efficient way to write code which is procedural, easy to reuse and maintain. Pig Latin also provides an way to express non-atomic data types, User Defined functions, variables to store relational operations.
The compilation of a Pig Latin program comprises of four stages \cite{gates2009building}: \begin{enumerate*}[label=\itshape\alph*\upshape)] \item parsing source Pig Latin program, program verification, type checking; \item generating logical plan; \item transformating into a physical plan; \item generating map-reduce jobs to run on Hadoop clusters \end{enumerate*}.

During the first step of Pig compilation, the parser verifies the syntactic correction of the program, type checking and schema inference etc. In the next phase, the parser generates a logical plan of the program that is actually a one to one mapping of the data transformation operations. In third stage each node in logical plan gets translated to one or a series of physical operators which in turn gets assigned to Hadoop stages.

Correct Pig compilation depends upon a non-trivial mapping from properties over Pig semantics to properties over MapReduce semantics through this sequence of transformation through the Logical Plan and Physical Plan itermediate representations. In each of these compilation phases, there is a possibility that a compiler implementation is in error, which would of course lead to incorrectness of the programs themselves.

When beginning this project, our goal was to formalize the semantics of the logical plan, physical plan, and MapReduce and to turn our informal ideas of compiler formalization into formal correctness proofs over well-defined semantics. This was an overly ambitious goal, and our work has not yet achieved it. However, we believe that we have made some important first steps
towards such a goal.

In particular, we believe that there is promise in our use of Coq to formulate Pig program specifications. As we will discuss in greater detail later in this report, we have set out to use Coq to define a formal model in which every Pig source program implies a single specification---a set of constraints---over the set of relations (a.k.a. tables) refered to in that program.

A proof of correctness of a program would then be a proof that when compiled to another form, the semantics of the latter form imply that these constraints are satisfied. A proof of compiler correctness is thus a proof that for any valid Pig program, a compiled form must satisfy that program's specification.

\textbf{Purpose of Report:} The goal of this report is to describe our formalizations, to discuss some of our design choices which we made when using the Coq proof assistant to implement our representation, and to note some of the challenges which we faced while using Coq to achieve these goals.

\textbf{Contributions:} The techinical contributions of our work are:

\begin{enumerate}
  \item The design of core calculus which we believe to be amenable for use
	      in proofs but also representative of five of the language's core features:
	\begin{enumerate}
		\item Programs specify queries over relations using common data
		      transformation operators.
		\item Programs manipulate aggregate non-atomic types (e.g. bags).
		\item Programs use user defined functions (UDFs) from an environment.
		\item Programs specify data flow via statements.
		\item Programs are parallelizable and distributable.
	\end{enumerate}
  \item The definition of Coq types to formalize the Pig ``logical plan'' and
	      a simple type system for it.
  \item The definition of Coq types to formalize the Pig ``physical plan''
	      and a simple type system for it.
  \item The definition of Coq types for instances of schema and relations.
\end{enumerate}

\textbf{Outline} The rest of the paper is organized as follows.

In the next section we provide some examples to show how relational operations are defined in Pig Latin programs and how those data transformations are converted to equivalent Logical and Physical plans, and to also give a sense  of how specifications may be derived from these.

In Sections 3 and 4, that we discuss some syntax and semantics of Pig Latin program followed by our discussion of their typing relations. In Sections 5 and 6, we discuss how specifications could be constructed from source programs and explain the basic idea behind how they could be used.

Sections 7-9 discuss related work and future work.
