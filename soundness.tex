\section{Meta-theory}
\label{sec:soundness}

We have not yet proven the standard soundness properties, since our formalization does not yet include traditional operational semantics.

We instead offer our initial work attempting to use Coq's powerful type system to define denotational semantics for Pig programs. We begin by constructing a type to represent all possible relation instances for a particular schema.

We follow the example of \cite{malecha2010rdbms} and use the term \texttt{support} to denote the type of a relation's row. The exact type of a row depends upon the relation's schema. We use a type-generating function to recursively define a particular support type given a particular schema.

\begin{lstlisting}
 Fixpoint support (s: schema_ty) : Type :=
  match s with
  | STyNil => unit
  | a *** => helper a
  | a *** s1 => (helper a) * support s1
  end
 with helper(c: col_ty) : Type :=
  match c with
  | CTyNat => nat
  | CTyBag s_nested => multiset (support s_nested)
  end.
\end{lstlisting}

For example, if a schema is just a single a column of \texttt{CTyNat}, then this function returns \texttt{nat}. As another example, if given the more complex schema, \texttt{CTyNat *** (CTyBag CTyNat)} (i.e. a nat column followed by a column of nat-bags), then this function returns the pair type \texttt{nat * (multiset nat)}.

In this way, we can use this function to construct for any schema a type which represents any possible instance of a row which might exist in any possible relation of that schema. This modeling should be useful in later proofs.

We then build on this definition to construct from a schema a type which represents all possible instances of a relation of that schema. We the Coq standard library's definition of a multiset to model this (since Pig uses multiset not set semantics for its relations).

\begin{lstlisting}
 Inductive relation (s: schema_ty) : Type :=
  | Relation: multiset (support s) -> relation s.
\end{lstlisting}

Essentially, \texttt{relation} is a type constructor, parameterized on a schema. It gives a way of constructing different types which ``box'' a \texttt{multiset} with an appropriate support.

Using our two examples from before:

\begin{itemize}
  \item The type \texttt{relation CTyNat} is the type which simply boxes an
        instance of \texttt{multiset nat}.
  \item The type \texttt{relation (CTyNat *** (CTyBag CTyNat))} is the type
        which boxes an instance of \texttt{multiset (nat * (multiset nat))}.
\end{itemize}

Each of these is a type which represents all possible instances of a relation with a particular schema. The type constructor \texttt{relation} itself is a way of constructing a type representing all possible relations of any possible schema.

We now show our initial understanding of how one might take these definitions and use them to construct powerful specification propositions (i.e. statments to-be-proven).

\textbf{TODO: Explain the next listing.}

\begin{lstlisting}
 Definition is_filtered (s:schema_ty) (r0: relation s) (f:filter s)
  (r1: relation s) :=
  let m0  := (relation_data s r0) in
  let m1 := (relation_data s r1) in
  forall tup: (support s),
    multiplicity m0 tup = match (f tup) with
                         | true => multiplicity m1 tup
                         | false => 0
                         end.

 Inductive filtered (s: schema_ty):
  relation s -> filter s -> relation s -> Prop :=
  | Filtered : forall (f: filter s) (r0 r1: relation s),
     is_filtered s r0 f r1 ->
     filtered s r0 f r1.
\end{lstlisting}

\textbf{TODO: Explain how these specifications can (trivially and statically) be extracted from the source.}

\textbf{TODO: Explain how this can be used to provide proofs.}

\textbf{TODO: Explain potential problems with using multiset.}
